\documentclass{article}

\usepackage{listings}

\title{User Manual}
\author{Nicolas Bonfante}
\date{August 2015}

\begin{document}

\maketitle
\tableofcontents
\newpage

\section{Introduction}
This tool allows you to animate a SVG file with a VCD file which come from a simulation tool. There are a few steps to follow to make the tool working with your
files. But don't worry, it will be very easy.

\section{VCD file}
In order to generate your VCD file, you will probably use tools like GHDL. This tool is normally fully-compatible with GHDL and not tested with other tools (because
of the lack of time and of the price of the licence). 
What you have to do with your VCD file is to find the name of the variable and the module where the variable take place. Let take an example.
\begin{lstlisting}
$date
  Thu Jul  9 10:11:34 2015
$end
$version
  GHDL v0
$end
$timescale
  1 fs
$end
$var reg 8 ! resultat[7:0] $end
$var reg 1 " monreset $end
$scope module uut $end
$var reg 1 # reset $end
$upscope $end
\end{lstlisting}
If you want to animate the variable called "reset", you need to note somewhere (inside your brain, if you have one) "uut.reset".

\section{SVG file}
There are three things to do with the SVG file. There is no restrictions on the SVG file BUT you need to follow these steps.
\subsection{Create groups}
You need to create extra groups in order to make the tool working :
\subsubsection{1 bit variable}
\subsubsection{n bits variable}
\subsection{Add !!}
The SVG name is  the value of the tspan node in the group you just define. You MUST add '!' before and after the name. You must also note the SVG name for
the COR file.
\subsection{Add an attribute}
In order to say to the tool that this group is animated, you must add the attribute "f2d=yes" to the g node.
\subsection{Example}
Here is the SVG file for our example.
\begin{lstlisting}
<g
 f2d="yes"
 id="g3387">
<path
   transform="matrix(1.7426543,0,0,1.5321982,-493.07809,155.18924)"
   inkscape:connector-curvature="0"
   id="path2983-91"
   d="m 531.0877,203.58151 c 0.50507,-28.53681 0,-29.04188 0,-29.04188"
<text
   xml:space="preserve"
   x="424.87717"
   y="416.3732"
   id="text3113-9-5-9"
   sodipodi:linespacing="125%"><tspan
	 sodipodi:role="line"
	 id="tspan3115-1-2-2"
	 x="424.87717"
	 y="416.3732"
	 style="font-size:12px;fill:#280b0b">!X1!</tspan></text>
</g>
\end{lstlisting}

\section{COR file}
A COR file is a very simple file that make the correspondance between the SVG file and the VCD file. Each lign is a pair of a VCD name and a SVG name. The syntax for
a lign is the following :
\begin{center}
SVG\_name:VCD\_name
\end{center}
Here is the COR file for our example.
\begin{lstlisting}
rst:uut.reset
\end{lstlisting}

\section{Bug report}
If you found some bugs or if you want new functionalities, feel free to send me an e-mail. 

\end{document}
