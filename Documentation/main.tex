\documentclass[a4paper]{article}

\usepackage[english]{babel}
\usepackage[utf8]{inputenc}


\title{Introduction to VCD}
\author{Nicolas BONFANTE}
\date{\today}

\begin{document}

\maketitle

\begin{center}
\textit{The aim of this document is in the first part to learn more on VCD file. In a second part I will describe my project and few solutions. }
\end{center}

\newpage

\part{What is a VCD file ?}

\section{Description}
VCD stands for " Value Change Dump ". This is a format for dumpfile generated by simulation tools such a GHDL (the one I use). It use a quit easy but uncommon syntax.

\section{Syntax}

A VCD file has 4 main sections : 
\begin{itemize}
\item Header section 
\item Variables definitions section 
\item \$dumpvars section 
\item Values changes section 
\end{itemize}

Beware ! VCD is case sensitive.

\subsection{Header}
This section contains a timestamp, a simulator version number, and a timescale. You can also include some comments.
\subsection{Variables definitions}
This section contains variables definitions. We use the following syntax to define variables :
\begin{center}
\$var type bitwidth id name
\end{center}
The id is an ASCII-character from  ! to ~ .
\subsection{\$dumpvars}
This section only contains variables initializations i.e the values of the variables at time t=0.
The syntax is the following : 
\begin{center}
values id
\end{center}
without space except for variables having bitwidth of 8.
\subsection{Values changes}
We denote the time by the following syntax :
\begin{center}
\#time
\end{center}
and the fact that a variable changes by :
\begin{center}
new\_value id
\end{center}
without space.

\newpage
\part{Library}

\section{SVG}

\subsection{svg.js}
A little bit more complex than the others. There is also some animation functions but less than in snap.svg. 

\subsection{snap.svg}
This library is open source and free. The documentation is well-explained and comprehensive. There is a lot of examples. 
There is animation handler that looks good. We can use external SVG files : I mean, not generated with snap.svg. Used by Adobe. By the developers of raphael.js in order to support new browsers.

\subsection{raphael.js}
Consider enverything as a DOM object. There is animation functions too. Well documented. Most used for now.

\section{VCD}

\section{Tessel VCD parser}
The only javascript vcd parser. So the best one. 

\section{My choices}


\end{document}